\documentclass[a4paper,11pt]{article}
                               
\usepackage[paper=a4paper,left=30mm,width=150mm,top=25mm,bottom=25mm]{geometry}
\usepackage{natbib} % Use the natbib bibliography and citation 
\usepackage{amsmath}
\usepackage{amsfonts}
\usepackage{graphicx}
\usepackage[parfill]{parskip} % no indentation before  a new paragraph
\usepackage{url}


\title{Sports user modeling based on GPS data}
\author{}
\date{}

\newtheorem{prop}{Proposition}
\newtheorem{theorem}{Theorem}[section]
\newtheorem{lemma}[theorem]{Lemma}
\newtheorem{proposition}[theorem]{Proposition}
\newtheorem{corollary}[theorem]{Corollary}

\newenvironment{proof}[1][Proof]{\begin{trivlist}
\item[\hskip \labelsep {\bfseries #1}]}{\end{trivlist}}
\newenvironment{definition}[1][Definition]{\begin{trivlist}
\item[\hskip \labelsep {\bfseries #1}]}{\end{trivlist}}
\newenvironment{example}[1][Example]{\begin{trivlist}
\item[\hskip \labelsep {\bfseries #1}]}{\end{trivlist}}
\newenvironment{remark}[1][Remark]{\begin{trivlist}
\item[\hskip \labelsep {\bfseries #1}]}{\end{trivlist}}

\newcommand{\qed}{\nobreak \ifvmode \relax \else
      \ifdim\lastskip<1.5em \hskip-\lastskip
      \hskip1.5em plus0em minus0.5em \fi \nobreak
      \vrule height0.75em width0.5em depth0.25em\fi}

\begin{document}
\maketitle

%%%%%%%%%%
% Intro
%%%%%%%%%%

\section{Introduction}

Wearable is the new trend, the location data of the users has already flooded to us. Among the 
mobile applications, sports management APPs heavily depend on recoding the users positions
along their workouts. In this project, we try to answer the following questions:

\begin{itemize}
\item Is it possible to generate good users fingerprints (features) based on the telematics data? 
With the fingerprints, we could do many things such as users clustering, users recommendations.
It is important because it is the corner stones of further applications. The effectiveness of the 
fingerprint could be tested in a setting that using the fingerprint to differ different users. 
Besides directly compare users based on fingerprints, some basic machine learning models
utilize the fingerprints could be used to classify routes to users as the Kaggle challenge
 \footnote{\url{https://www.kaggle.com/c/axa-driver-telematics-
analysis/forums}\label{fn:1}}.

\item After the fingerprints generated, we could test using these fingerprints to do routes 
recommendation. We could model this problem as a structured prediction problem as shown in 
the next section.
\end{itemize}

A simple search in google scholar returned noting on the same topic. 

\section{Methods}

The Kaggle challenge tried to find good features and 
models to differ drivers based on their telematics
data. It provide us many good ideas,  hands on experience and valuable tips. 
Github repositories \footnote{\url{https://www.kaggle.com/c/axa-driver-telematics-analysis/forums/t/12849/github-repos-now-live}} of the attendants of the challenge will give us a boost in the early
development. 

\subsection{Feature extraction}

\paragraph{Routes geo features}
These features are about what kind of routes one may like. 
The features summaries the routes geographical characteristics.
Such as the total distance, number of bends. elevations, environments (sea side, forest, city...). 

\paragraph{Routes simplified geo features}
The GPS contains lots of noise, we could use RDP algorithm \footnote{\url{https://en.wikipedia.org/wiki/
Ramer?Douglas?Peucker_algorithm}} to simplify the routes and extract geo features again on the
simple version.

\paragraph{Route human features} These features are about in what ways, one like running or cycling
their routes. Such as average speed, histogram of speed, percentiles of speed, accelerations, etc.

\subsection{Fingerprint testing}
Let a trip $t$ contains route geo features $g$ and human features $h$, i.e., $t = (g,h)$. Assume
we have $n$ users indexed by $i$ and each user has $c_i$ trips indexed by $j$,  so the data is
represented as $((t_j)_{j=1}^{c_i})_{i=1}^n$, then

\begin{align*}
t_j \sim M_i, \; j = 1, \ldots, c_i 
\end{align*}

where $M_i$ is the model of the $i^{th}$ user. The simplest model could be a multi-dimensional
gaussian.  We could also using random forest, logistic regression or SVM which proved to be 
useful in the Kaggle challenge. The criteria of the a good model is to differ trips of different users 
with the model.

\subsection{Routes recommendation}

In this problem, we represent users by the human features and the trip by geo features,
i.e. $h \in \cal{H}$, $g \in \cal{G}$. We map the users, trips into a joint space $\cal{H} \otimes
\cal{G}$. Then for each user, we have the following model:

\begin{align}
\min_{\mathbf{w}} &\quad \frac{1}{2} ||\mathbf{w}||^2 \nonumber \\
s.t. &\quad \mathbf{w}^T(h_j \otimes g_j)  \geq \mathbf{w}^T(h_j\otimes g_t) \quad \text{for}\, j \in T_i, t \in T \setminus T_i,
\end{align}
where $T_i$ is the set of trips of $i^{th}$ user and $T$ is the set of trips not belong to $i^{th}$ user. 
When the $i^{th}$ user go to another (or the same) place, we first find all the trips $T_y$ from other 
people in the same area and represent the trips only by geo features $T_y = (g_t)_{t=1}^{|T_y|}$, then 
we rank the routes based on the scores (pre-image):

\begin{align}
\max_{g_t\in T_y} \sum_{j\in T_i}\mathbf{w}^T(h_j\otimes g_t).
\end{align}

The above can be done in kernel space if the number of trips is not very large.

\bibliographystyle{natbib}
\bibliography{plan}

\end{document}

















