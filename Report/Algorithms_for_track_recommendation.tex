
\section{Algorithm}

Most application records the walking, running and cycling routes after the 
exercise have been done. However, when user move to a new place or travel to
a new country, find a good running route will become a problem. In this
section, we adpat algorithms in Graph for route recommendation.

We consider a simple case where a user specify the length of the route and we
recommend a route that starts and ends at the same position with the length
as required. Formally, given a undirected graph $G = (V,E)$ and a starting 
vertex $s$, the goal is to find a walk in the graph $G$ of the length $d$ ends at $s$.
We have length information $w_e$ for every edge $e \in E$. In practice, it is 
possible in $G$ there is no walk of length $d$ starting from $s$ and ending at
$s$, thus we allow the maximum length difference $\Delta$. We may have a lot of tottering
in the end, so we sort in the reverse order the walks by the number of different nodes the
walk contains and take the first few ones. The simple BFS based algorithm is listed in
Algorithm \ref{alg:bfs1}.

\begin{algorithm}[H]
\begin{algorithmic}[1]
\STATE Input: $G=(V,E), s, d$
\STATE Output: $P$, the set of walks of length $d$ start and end at $s$
\STATE Init $P=\{\}, Q=\{\}$
\STATE $Q.add(s)$
\WHILE{$Q$ is not empty}
\STATE$u = Q.pop()$
\FOR{$v \in neighbors(u)$}
  \IF{$v == s$ and $depth(v)-d \leq \Delta$}
   \STATE $P.add(v)$
  \ELSIF{$depth(v) < d+\Delta$}
   \STATE $Q.add(v)$
  \ENDIF 
\ENDFOR
\ENDWHILE
\STATE Sort the walks in $P$ by the number of different nodes in reverse order.
\end{algorithmic}
\caption{Na\"ive BFS based search.}
\label{alg:bfs1}
\end{algorithm}

We can speed up the search by joining two walks into one. The idea is simple, when we have all
the nodes in the next layer of the search tree, we test for all pairs of nodes if they are the same.
For the nodes on different paths of the search tree, if they are the same, we can concatenate the two
paths into one, the detail is listed in Algorithm \ref{alg:bfs2}.

\begin{algorithm}[H]
\begin{algorithmic}[1]
\STATE Input: $G=(V,E), s, d$
\STATE Output: $P$, the set of walks of length $d$ start and end at $s$
\STATE Init $P=\{\}, Q=\{\}$
\STATE $Q.add(s)$
\WHILE{$Q$ is not empty}
\STATE $C=\{\}$
\FOR{$u \in Q$}
\FOR{$v \in neighbors(u)$}
\IF{$depth(v) \leq d+\Delta$}
\STATE $C.add(v)$
\STATE $Q.add(v)$
\ENDIF
\ENDFOR
\ENDFOR
\FOR{$u\in C$}
\FOR{$v\in C$}
\IF{$u==v$ and $depth(u)+depth(v)-d\leq\Delta$}
\STATE $P.add($Walk by joining $u$ and $v)$
\ENDIF
\ENDFOR
\ENDFOR
\ENDWHILE
\STATE Sort the walks in $P$ by the number of different nodes in reverse order.
\end{algorithmic}
\caption{Na\"ive BFS based search.}
\label{alg:bfs2}
\end{algorithm}

One of the problem for the above algorithms are the computational efficiency. The problem of finding a path
of fixed length in a graph is NP-hard. Since each of the blocks in our data is about 10m to 20m, to find a 
route of 3km needs 150 to 300 blocks and the search tree is huge. In practice, we could only find route
round 1km in reasonable time. As a result, to find longer route, we searched the historical routes around 
the area for every user and suggest the most similar one in terms of length.

Based on historical routes datasets, we could get the information about the number of visits of
every places and take that as popularity. This open the possibility to recommend routes with most
popularity. We add this function as a post-processing of the routes in the set $P$ which is just sorting
based on the sum of number of visits of every places in a route. We show such an route at
\url{http://ch3.strikingly.com/blog/recommend-route-with-popularity}.

We also retrieved twitter data related to places in the routes. By doing so, we open the possibility to add
social information and even sentimental information in the route recommendation. We show such an route
at \url{http://ch3.strikingly.com/blog/routes-with-social-info}.














