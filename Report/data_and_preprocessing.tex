
\section{Data preprocessing}

\subsection{Preliminaries}

\edmd\ dataset contains the information of a collection of $68479$ sport tracks (e.g., walking, running, cycling).
We use $\Tcal=\{T_i\}_{i=1}^n$ to denote the collection of tracks where we have $n=68479$.
Each track $T_i\in\Tcal$ is represented as a set of geographical points on the map (\gps\ locations) ordered by timestamps defined as
\begin{align*}
	T_i = \{(x_{i,1},y_{i,1},t_{i,1}),\cdots,(x_{i,m_i},y_{i,m_i},t_{i,m_i})\},
\end{align*}
where $x_{i,j}$ is the longitude, $y_{i,j}$ is the latitude, $t_{i,j}$ is the timestamp when $(x_{i,j},y_{i,j})$ is recorded by the tracking devices, $m_i$ is the total number of pointed recored for track $T_i$.
In addition, we have $t_{i,j}<t_{i,k}$ when $j<k$.

Each track $T_i$ is then represented as an undirected graph $G_i = (V_i,E_i)$ defined on a $2$D surface.
The vertex set $V_i=\{(x_{i,j},y_{i,j})\}_{j=1}^{m_i}$ includes all \gps\ locations of track $T_i$.
There exist an undirected edge $e_{i,k}$ between vertex $v_{i,k}$ and $v_{i,k+1}$ $\forall k\le m_i-1$.
As a result, the edge set is defined as a collection of undirected edges $E_i = \{e_{i,k}\}_{k=1}^{m_i-1}$.

\subsection{Global track graph}\label{build_global_graph}

The goal is to analyze all available track information and recommend an ideal track for individual end user.
We realize that it is feasible to maintain all individual track information $T_i$ and $V_i$ due to the following reasons
\begin{itemize}
	\item It is computationally expensive to analyze $68479$ tracks each time when we perform the recommendation or searching algorithm. In particular, the data files for all tracks take approximately $2$ Gigabyte.
	\item It is difficult to measure global performances along the tracks (e.g., average/min/max speed on a location, number people running over a location) by analyzing track information separately. 
	\item Individual track information is not an accurate representation of the track due to the nature of the \gps\ devices and the tracking algorithms. In other word, several tracks might be performed on the same street but have different trajectories in terms of \gps\ location data. 
\end{itemize} 

We summerize and represent all track information via a global track graph $\gtt=(\vtt,\ett)$.
\begin{itemize}
	\item The world map is partitioned into small blocks of $10m\times10m$. Each block is represented by the \gps\ location $(x,y)$ of its centre, and is a potential vertex in the global track graph $\vttt=(x,y)\in\vtt$. 
	\item Then we map the set of vertices $v_{i,j}\in V_i$ in the graph $G_i$ of track $T_i$ to $\vtt$ by assigning $v_{i,j}$ to the nearest vertex $\vttt\in\vtt$. Essentially, each vertex $\vttt$ corresponds to a collection of vertices from different tracks
	\begin{align*}
		\vttt_k = \{v_{i,j}\}_{i\in\{1,\cdots,n\}, j\in\{1,\cdots,m_i\}}, k\in\{1,\cdots,|\vtt|\}
	\end{align*}
	\item There exists an undirected edge $\ettt=(\vttt_p,\vttt_q)\in\ett$ if there is an edge between $v_p\in\vttt_p$ and $v_q\in\vttt_q$. Mathematically, the edge set $\ett$ can be defined as
	\begin{align*}
		\ett = \{(\vttt_p,\vttt_q)|\exists (v_p,v_q)\in E_i, \forall v_p\in\vttt_p,v_q\in\vttt_q, i\in\{1,\cdots,n\},p,q\in\{1,\cdots,|\vtt|\}\}
	\end{align*}
\end{itemize}

As a result, we have a global track graph $\gtt=(\ett,\vtt)$ which summerizes all track information and underlies all possible tracks that can be run by users.
The global track graph is a high level abstraction of the data and serves as the input to the recommendation/searching algorithms described in the next section. 

\subsection{Annotations on the global track graph}

We notice that the vertices in the global track graph are essentially \gps\ locations which not only can be annotated with the summary statistics computed from all track information but also can be annotated with enriched contextual information. 
Currently, we consider the following annotations on the vertices $\vtt$ of the global track graph $\gtt$
\begin{itemize}
	\item We compute various summary statistics from all track information, which include e.g., maximum speed on the current location, minimum speed of the current location, average speed of the current location, popularity of the current location in terms of the number of tracks going through it.
	\item For each location, we extract the information from social network (\twitter) and annotate the vertex with bag-of-words features.
\end{itemize}

\subsubsection{\twitter information}

Just like a community is characterized by its residents living in that area, a route is definitely can be defined by the persons running, living or walking along it.
Along the development of digitalization of life, people would like to share their feelings or experiences through media social tools, such as facebook and twitter.
According to the locations of some points in a route, we can get the information of twitters and the corresponding tweets sent from nearby.
Therefore, when users would like to browser routes in a certain region or search a routes under some constrains, these social information can be displayed along these routes.
In one hand, people can learn more about the routes instead of only from its geographic information but also maybe some feedback about that region.
On other hand, furthermore, people would find some potential exercise company or even turn this kind of virtual friend into friend in reality.
It may enrich their exercise events meanwhile broaden their social networking.

In the future, it seems can even add some tweets tag on the route allowing the user to follow the twitter user that he/she interets. Another possible way it to emble some other social elements into that map and even make it a potential business benefits for those company along the routes in city, especially for these new stores, they can easiy to advertise.


\subsection{Technical details}

Three cities are selected for demo (London, New York, and Amsterdam) by defining the city area as a block of $20km\times20km$ and using the \gps\ location data of the city centre as the block centre.
All tracks within the predefined city block are included in the analysis.
We following the procedure described in Section~\ref{build_global_graph} to build global track graphs.
The statistics of the graphs are shown in Table~\ref{graph_statistics}.

\begin{table}
	\begin{center}
		\begin{tabular}{|c||c|c|}\hline
			City & Vertices & Edges  \\ \hline \hline
			London & 45942 & 46182\\ \hline
			New York & 4856 & 6543\\ \hline
			Amsterdam & 12357 & 15862 \\ \hline
		\end{tabular}
		\caption{Statistics of the global track graphs.}
		\label{graph_statistics}
	\end{center}
\end{table}










